% !TEX TS-program = pdflatex
% !TEX encoding = UTF-8 Unicode

% This is a simple template for a LaTeX document using the "article" class.
% See "book", "report", "letter" for other types of document.

\documentclass[12pt]{book} % use larger type; default would be 10pt

\usepackage[utf8]{inputenc} % set input encoding (not needed with XeLaTeX)

%%% Examples of Article customizations
% These packages are optional, depending whether you want the features they provide.
% See the LaTeX Companion or other references for full information.

%%% PAGE DIMENSIONS
\usepackage{geometry} % to change the page dimensions
\geometry{a4paper} % or letterpaper (US) or a5paper or....
% \geometry{margin=2in} % for example, change the margins to 2 inches all round
% \geometry{landscape} % set up the page for landscape
%   read geometry.pdf for detailed page layout information

\usepackage{graphicx} % support the \includegraphics command and options

% \usepackage[parfill]{parskip} % Activate to begin paragraphs with an empty line rather than an indent

%%% PACKAGES
\usepackage{booktabs} % for much better looking tables
\usepackage{array} % for better arrays (eg matrices) in maths
%\usepackage{paralist} % very flexible & customisable lists (eg. enumerate/itemize, etc.)
\usepackage{verbatim} % adds environment for commenting out blocks of text & for better verbatim
\usepackage{subfig} % make it possible to include more than one captioned figure/table in a single float
% These packages are all incorporated in the memoir class to one degree or another...

%%% HEADERS & FOOTERS
\usepackage{fancyhdr} % This should be set AFTER setting up the page geometry
\pagestyle{fancy} % options: empty , plain , fancy
\renewcommand{\headrulewidth}{0pt} % customise the layout...
\lhead{}\chead{}\rhead{}
\lfoot{}\cfoot{\thepage}\rfoot{}

%%% SECTION TITLE APPEARANCE
\usepackage{sectsty}
\allsectionsfont{\sffamily\mdseries\upshape} % (See the fntguide.pdf for font help)
% (This matches ConTeXt defaults)

%%% ToC (table of contents) APPEARANCE
\usepackage[nottoc,notlof,notlot]{tocbibind} % Put the bibliography in the ToC
\usepackage[titles,subfigure]{tocloft} % Alter the style of the Table of Contents
\renewcommand{\cftsecfont}{\rmfamily\mdseries\upshape}
\renewcommand{\cftsecpagefont}{\rmfamily\mdseries\upshape} % No bold!

%%% END Article customizations

\usepackage[spanish]{babel}
\usepackage{listings} 
%%% The "real" document content comes below...

\title{Investigación del Lenguaje - Java}
\author{Adriana Rodríguez \and Marcelo Sánchez \and Raquel Villón}
%\date{} % Activate to display a given date or no date (if empty),
         % otherwise the current date is printed 

\begin{document}

\maketitle

\tableofcontents
\newpage
\mbox{}

\chapter{Introducción}
\paragraph{En este primer capítulo se podrá analizar de acerca de qué es Java, conocer acerca de las principales funcionalidades así como también las características que destaca a este lenguaje orientado a objetos.}
\section{?`Qu\'e es Java?}
\chapter{Características}
\chapter{Historia}
\paragraph{Según los hermanos Deitel (2007), el lenguaje java fue creado en 1991 por la compañía Sun Microsystems, la cual es una empresa dedicada a la fabricación y venta de servidores y componentes de componentes informáticos. Empezó como un proyecto de investigación de la compañía denominado Green y fue pensado para programar los dispositivos electrónicos inteligentes.}
\paragraph{Dietel también indica que Java está basado en C++. Al principio su nombre fue Oak, pero luego se descubrió que ese nombre ya lo tenía otro lenguaje de programación. Luego, en una reunión de la gente de Sun en una cafetería, se propuso el nombre de Java, el cual es una variedad de café. }
\paragraph{Al principio el proyecto Green tuvo problemas debido a que el avance de los dispositivos electrónicos inteligentes no se desarrollaba tan rápido como Sun había previsto. Sin embargo, cuando la World Wide Web explotó, Sun vio rápidamente el potencial que Java tenía para darle dinamismo a las páginas web. De esa manera el proyecto Green pudo avanzar y finalizar con una conferencia en mayo de 1995, en la que Sun Microsystems anunció formalmente la existencia del lenguaje de programación Java. }
\paragraph{A medida que la World Wide Web avanzaba Java tenía mayor acogida y se desarrolló de tal modo en el que luego se lo utilizó para desarrollar grandes aplicaciones empresariales, aplicaciones para dispositivos móviles, radiolocalizadores, asistentes personales digitales, entre otros. }
\paragraph{En el 2009, Sun Microsystems fue comprada por Oracle por un valor de 7.400 millones de dólares según el portal 20minutos.es (2009) Y actualmente se distribuye la versión número 7 de Java, así como también sus diferentes líneas. Estas líneas se especializan en un ambiente, por ejemplo: Java SE está creada especialmente para aplicaciones de escritorio; Java EE es especializado en crear aplicaciones web; Java ME es utilizado para desarrollar aplicaciones que se ejecutan en dispositivos móviles, etc. }



\chapter{Tutorial de Instalación}
\paragraph{Para poder correr aplicaciones creadas en Java se necesita cómo mínimo el Java Runtime Enviroment (JRE). Pero si lo que se quiere es desarrollar aplicaciones en Java se necesita el Java Development Kit (JDK) que contiene al JRE y herramientas para desarrollar, depurar y monitorear aplicaciones en Java (Oracle.com).}
\paragraph{En este documento mostraremos la instalación de Java SE 7, distribución para el desarrollo de aplicaciones de escritorio, en plataforma Windows 8 de 64bits.}
\paragraph{Podemos descargar El JDK directamente de la página oficial de Oracle, http://www.oracle.com/technetwork/java/javase/downloads/index.html.}
\paragraph{•}
\paragraph{•}

\chapter{Hola Mundo y otros Programas Introductorios}
\lstset{language=Pascal}          % Set your language (you can change the language for each code-block optionally)
\begin{lstlisting}[frame=single]  % Start your code-block
for i:=maxint to 0 do
begin
{ do nothing }
end;
Write('Case insensitive ');
Write('Pascal keywords.');
\end{lstlisting}

\begin{thebibliography}{99}
\bibitem{ob} Belmonte, O. (2005). Introducción al lenguaje de programación Java. [en línea]. Disponible en: 'http://www3.uji.es/belfern/pdidoc/IX26/Documentos/introJava.pdf' [2013, 19 de Octubre].
\end{thebibliography}

\end{document}
